\begin{foreignabstract}

This work consists in an analysis of nonlinear system identification methods applied to digital audio signals, focused on comparatively evaluating the researched algorithms in terms of efficiency and robustness. Nonlinear system identification is a pertinent problem to many branches of engineering, but the most popular methods are often costly and/or limited. Given the scarcity of more complex techniques, alongside real world applications to audio signals, such as the automatic extraction of the soundtrack of an audiovisual production, it is worthwhile to explore new horizons of this research field. Assuming that we have the original signal and its (possibly noisy) desired distorted version, three models were considered: the Wiener Filter, used to assess the performance of a linear method on the estimation of nonlinear distortions; the Correntropy Filter, that uses kernel functions to imitate nonlinear effects; and the Unscented Kalman Filter, that applies the unscented transform for a better identification process. All methods had to be adapted to be used. Then, the adapted methods were used to try to reproduce distortions in (artificially generated) audio recordings, with or without noise. The results showed that: the linear model could not simulate nonlinearities, but it had the best noise removal performance; the Correntropy Filter exhibited poor estimations, mainly because of the strategy adopted; and the unscented model was able to reproduce the distortions, but it was less robust to noise than the Wiener Filter.

\vspace*{7mm}
\noindent \textit{Keywords:} audio signal processing, audio source separation, nonlinear distortion, additive noise, Wiener Filter, Kalman Filter, correntropy.

\end{foreignabstract}
