\begin{abstract}

Este trabalho trata da análise de métodos de identificação de sistemas não-lineares aplicados a sinais de áudio digitais, visando a avaliar comparativamente os algoritmos estudados quanto a eficiência e robustez. A identificação de sistemas não-lineares é um problema relevante para vários ramos da engenharia, mas os métodos mais famosos costumam ser custosos e/ou limitados. Diante da escassez de técnicas mais complexas, somada às aplicações reais no contexto de áudio, como a remoção automática da trilha sonora de uma produção audiovisual, torna-se pertinente explorar novos horizontes desta área de pesquisa. Pressupondo a presença de um sinal original e sua versão desejada (e possivelmente ruidosa), foram considerados três modelos: o Filtro de Wiener, para avaliar a eficiência de um método linear na estimação de distorções não-lineares; o Filtro de Correntropia, que usa funções núcleo para tentar mimetizar efeitos não-lineares; e o Filtro de Kalman \textit{Unscented}, que utiliza a transformação \textit{unscented} para um melhor processo de identificação. Todos os filtros sofreram adaptações para serem usados e foram avaliados no âmbito da reprodução de distorções em gravações sonoras (geradas artificialmente), com ou sem ruído. Os resultados demonstraram que: o modelo linear não conseguiu reproduzir não-linearidades, mas foi o que melhor removeu sinais espúrios; o Filtro de Correntropia não apresentou resultados satisfatórios, principalmente por causa da estratégia empregada; e o modelo \textit{unscented} conseguiu reproduzir as distorções desejadas, mas foi menos robusto a ruído se comparado ao Filtro de Wiener.

\vspace*{5mm}
\noindent \textit{Palavras-chave:} processamento de sinais de áudio, separação de fontes sonoras, distorção não-linear, ruído aditivo, Filtro de Wiener, Filtro de Kalman, correntropia.

\end{abstract}
