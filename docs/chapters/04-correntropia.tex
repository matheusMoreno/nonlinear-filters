\chapter{Filtro de Correntropia}
\label{chapter:correntropyfilter}

Agora que sabemos das capacidades e limitações --- dentro do contexto do projeto --- de
um método de identificação linear consolidado, podemos partir para técnicas inovadoras
desenvolvidas especificamente para sistemas não-lineares. A primeira a ser abordada é o
Filtro de Correntropia~\cite{pokharel-2006, pokharel-2007}. Baseado em modelos que
utilizam funções núcleo (\textit{kernel functions}), o filtro se propõe a atacar o
\textit{trade-off} entre complexidade e robustez por meio de uma função denominada de
correntropia~\cite{santamaria-2006}. O método é também chamado de Filtro de Wiener
Não-linear, pois o Filtro de Wiener FIR discreto é usado em seu desenvolvimento, como
será visto em breve.

Começaremos o capítulo revisando o conceito de funções núcleo, para que em seguida seja
possível introduzir a função de correntropia. Então, o Filtro de Correntropia será
formalmente deduzido. Além disso, será apresentada também uma forma matricial da
solução, a qual foi desenvolvida durante o projeto; com ela, pôde-se implementar o
algoritmo de forma vetorizada, reduzindo drasticamente seu tempo de execução. Em
seguida, uma análise preliminar acerca do filtro será conduzida. Por fim, o modelo
específico utilizado no trabalho será discutido, seguido pelos experimentos executados
e seus respectivos resultados.

\section{Funções núcleo}

Antes de definirmos o que é a correntropia em si, devemos considerar o tópico mais
amplo de funções núcleo. Popularizada por suas inúmeras aplicações na área de
aprendizado de máquina, essa classe de funções envolve noções fundamentais de análise
funcional e espaços de Hilbert em suas definições e propriedades. Ainda assim, é
possível desenvolver uma discussão suficientemente ampla sobre o tema seguindo uma
abordagem mais intuitiva e abstraindo-se dos detalhes técnicos. Este será o caminho que
adotaremos pelo resto da seção, já que se aprofundar nessa teoria vai além do escopo do
projeto. O leitor interessado encontra em~\cite{principe-2010} uma discussão mais
completa acerca do assunto.

Essencialmente, uma função núcleo computa a \emph{similaridade} entre dois vetores
$\mathbf{x}_1$ e $\mathbf{x}_2$ presentes no mesmo espaço vetorial $X$. Porém, ao
contrário de outros operadores que visam ao mesmo objetivo --- como a norma euclidiana
da diferença ou a similaridade por cosseno ---, a função implicitamente mapeia os
pontos para um espaço vetorial de maior dimensão $F$, e então calcula o produto interno
dos vetores transformados. A ideia por trás desse mapeamento é gerar dimensões que
reflitam outros aspectos sobre os dados, e, assim, fazer com que o produto interno
entre os pontos mapeados seja mais elucidativo em relação à similaridade entre
$\mathbf{x}_1$ e $\mathbf{x}_2$.

Desse modo, uma \emph{função núcleo} $\kappa$ é definida~\cite{shawetaylor-2004} como
sendo uma função que, para todos $\mathbf{x}_1, \mathbf{x}_2 \in X$, satisfaz
\begin{equation}
	\kappa(\mathbf{x}_1, \mathbf{x}_2) = \bm{\phi}(\mathbf{x}_1)^T \bm{\phi}(\mathbf{x}_2),
	\symbl{$\kappa$}{Função núcleo}
	\symbl{$\bm{\phi}$}{Transformação vetorial associada à função núcleo}
\end{equation}
onde $\bm{\phi}$ é uma função vetorial definida por
\begin{equation}
	\bm{\phi} : \mathbf{x} \in X \subseteq \mathbb{R}^p \longmapsto \bm{\phi}(\mathbf{x}) \in F \subseteq \mathbb{R}^P.
\end{equation}
Para que a estratégia faça sentido, $P \geq p$. É importante notar que \emph{não é preciso conhecer o mapeamento $\bm{\phi}$ associado à função núcleo.} Na verdade, a força do método é justamente essa: usando um núcleo válido\footnote{Um ``núcleo válido'', ou núcleo de Mercer~\cite{principe-2010}, é contínuo, simétrico e positivo definido. Note que, por causa da simetria, podemos afirmar que $\kappa(\mathbf{x}_1, \mathbf{x}_2) = \kappa(\mathbf{x}_2, \mathbf{x}_1)$.}, calcula-se diretamente o produto interno entre dois vetores em $F$, sem aplicar a função $\bm{\phi}$.

Vamos ilustrar a capacidade das funções núcleo por meio de um exemplo. Considere os
vetores $\mathbf{x}_1, \mathbf{x}_2 \in \mathbb{R}^2$ descritos por
\begin{equation}
	\mathbf{x}_1 = \begin{bmatrix} \alpha_1\\\beta_1 \end{bmatrix},\ \mathbf{x}_2 = \begin{bmatrix} \alpha_2\\\beta_2 \end{bmatrix}.
\end{equation}

Para calcularmos a semelhança entre os vetores, em vez de usarmos o produto interno
convencional, usaremos o núcleo polinomial~\cite{principe-2010} de segunda ordem
$\kappa(\mathbf{x}_1, \mathbf{x}_2) = ( \mathbf{x}_1^T \mathbf{x}_2 + 1)^2$. Assim,
\begin{align}
	\kappa(\mathbf{x}_1, \mathbf{x}_2) & = (\alpha_1 \alpha_2 + \beta_1 \beta_2 + 1)^2                                                                                     \\
	                                   & = \alpha_1^2 \alpha_2^2 + 2 \alpha_1 \alpha_2 + 2 \alpha_1 \alpha_2 \beta_1 \beta_2 + 2 \beta_1 \beta_2 + \beta_1^2 \beta_2^2 + 1 \\
	                                   & = \bm{\phi}(\mathbf{x}_1)^T \bm{\phi}(\mathbf{x}_2);
\end{align}
nesse caso, a função $\bm{\phi}$ implicitamente aplicada pelo núcleo foi
\begin{equation}
	\bm{\phi}(\mathbf{x}_i) = \begin{bmatrix}
		\alpha_i^2              &
		\sqrt{2}\alpha_i        &
		\sqrt{2}\alpha_i\beta_i &
		\sqrt{2}\beta_i         &
		\beta_i^2               &
		1
	\end{bmatrix}^T.
\end{equation}

Note que $\bm{\phi}(\mathbf{x}_i)$ contém mais informações do que $\mathbf{x}_i$,
embora os dados de origem sejam os mesmos. Com $\bm{\phi}(\mathbf{x}_i)$, poderíamos,
por exemplo, modelar um sistema quadrático com um algoritmo linear, visto que os termos
quadráticos já estão presentes no vetor de entrada. Em outras palavras, \emph{certas
	relações não-lineares em $X$ se tornam lineares em $F$}. Podemos generalizar essa
estratégia para qualquer tipo de núcleo, com a restrição de que normalmente teremos
apenas $\kappa$, exigindo então que o algoritmo trabalhe não com vetores individuais,
mas com seus produtos internos.

Uma das funções núcleo mais importantes --- quiçá a mais importante --- é a gaussiana
invariante à translação~\cite{santamaria-2006}, definida como sendo\footnote{A função é
	normalmente definida sem o fator multiplicativo $1/\sqrt{2\pi}\sigma$, mas como
	em~\cite{santamaria-2006} o termo está presente, ele também se encontra na nossa
	definição.}
\begin{equation}
	\kappa(\mathbf{x}_1, \mathbf{x}_2) = \frac{1}{\sqrt{2\pi} \sigma} \exp\left( - \frac{\lVert \mathbf{x}_1 - \mathbf{x}_2 \rVert^2}{2\sigma^2} \right),\symbl{$\sigma$}{Parâmetro do núcleo gaussiano}
\end{equation}
onde $\sigma$ é um parâmetro ajustável. A principal característica deste núcleo é que seu mapeamento $\bm{\phi}$ associado gera um espaço $F$ de dimensão infinita, o que torna computacionalmente impossível calcular $\bm{\phi}(\mathbf{x})$ explicitamente. Além disso, algumas outras particularidades desse núcleo o fazem ser usado na função de correntropia, como veremos a seguir.

Em suma, núcleos são usados para calcular a semelhança entre dois vetores considerando
transformações não-lineares aplicadas sobre seus elementos. Com eles, podemos tentar
fazer com que relações não-lineares em um espaço vetorial se tornem lineares em outro,
o que possibilita a aplicação de métodos mais simples na solução de problemas. Embora
poderosas, essas funções não podem ser usadas em qualquer algoritmo linear; é preciso
que este trabalhe com produtos internos para que, assim, as ocorrências de
$\mathbf{x}_1^T\mathbf{x}_2$ sejam substituídas por um núcleo $\kappa(\mathbf{x}_1,
	\mathbf{x}_2)$ apropriado.

\section{Função de correntropia}

Agora que possuímos um melhor entendimento sobre funções núcleo, a função de
correntropia pode ser formalmente apresentada.

Seja $\{\mathbf{x}[n]\}$ uma sequência estocástica, com $n$ sendo o índice associado ao
elemento do conjunto, e $\mathbf{x}[n] \in \mathbb{R}^p$, $p \in \mathbb{N}^*$. A
função de correlação generalizada, ou simplesmente \emph{correntropia}, é uma função $V
	: \mathbb{R}^p \times \mathbb{R}^p \mapsto \mathbb{R}$ definida~\cite{santamaria-2006}
por
\begin{equation}
	V_{\textbf{x}\textbf{x}}[n_1, n_2] = E\{ \kappa(\mathbf{x}[n_1], \mathbf{x}[n_2]) \},
	\symbl{$V_{\cdot\cdot}{[k]}$}{Função de correntropia}
\end{equation}
onde $\kappa$ é um núcleo válido e $E\{\cdot\}$ é o operador de valor esperado estatístico. Como tanto em~\cite{santamaria-2006} quanto em~\cite{pokharel-2006} o núcleo utilizado é o gaussiano, este também será o qual usaremos pelo restante do capítulo. É importante mencionar que, assim como a função núcleo utilizada para sua concepção --- independente de qual seja, contanto que seja válida ---, a correntropia é uma função simétrica positiva definida.

O que faz a função de correntropia ser tão interessante? Em primeiro lugar, a partir da
discussão sobre núcleos da seção anterior, podemos afirmar que a correntropia calcula a
similaridade média entre dois instantes de uma sequência considerando complexas
características não-lineares. Além disso, se expandirmos o núcleo gaussiano em uma
série de potências, encontramos
\begin{equation}
	V_{\textbf{x}\textbf{x}}[n_1, n_2] = \frac{1}{\sqrt{2\pi}\sigma} \sum_{i=0}^\infty \frac{(-1)^i}{2^i \sigma^{2i} i!} E\{ \lVert \mathbf{x}[n_1] - \mathbf{x}[n_2] \rVert^{2i} \}.
	\label{eq:correntropy:series-expansion}
\end{equation}
Note que todos os momentos pares da variável aleatória $\lVert\mathbf{x}[n_1] - \mathbf{x}[n_2]\rVert$ estão presentes na série. Mais especificamente, o termo em $i=1$ é proporcional a
\begin{equation}
	E\{ \lVert \mathbf{x}[n_1] - \mathbf{x}[n_2] \rVert^2 \} = E\{ \lVert \mathbf{x}[n_1] \rVert^2 \} + E\{ \lVert \mathbf{x}[n_2] \rVert ^2 \} - 2 E\{ \mathbf{x}[n_1]^T \mathbf{x}[n_2]\},
	\label{eq:correntropy:expansion-1st-term}
\end{equation}
e, como $E\{ \mathbf{x}[n_1]^T \mathbf{x}[n_2] \} = R_{\mathbf{x}\mathbf{x}}[n_1, n_2]$, podemos concluir que as informações associadas à autocorrelação (ou correlação cruzada) estão presentes na função. Ademais, se a sequência tiver média zero, a Equação~\eqref{eq:correntropy:expansion-1st-term} demonstra que a função também incluirá informações acerca das variâncias nos instantes considerados (e a autocorrelação se tornará a covariância).

Ainda analisando a Equação~\eqref{eq:correntropy:series-expansion}, é possível concluir
que, para que a correntropia dependa apenas do intervalo entre $n_1$ e $n_2$, é preciso
que todos os momentos pares de $\lVert\mathbf{x}[n_1] - \mathbf{x}[n_2]\rVert$ sejam
constantes. Uma condição suficiente para isso é que a sequência aleatória
$\{\mathbf{x}[n]\}$ seja estacionária no sentido estrito (SSS, de \textit{Strict-Sense
	Stationary}\abbrev{SSS}{\textit{Strict-Sense Stationary}})~\cite{peebles-1987}. Deste
modo, $V_{\textbf{x}\textbf{x}}$ dependerá apenas do intervalo $k = n_2 - n_1$. A
função $V_{\textbf{x}\textbf{x}}[k]$ é positiva, simétrica, e com valor máximo na
origem.

Supondo a condição de estacionariedade estrita, a correntropia de uma sequência finita
de $N$ amostras pode ser estimada por meio da média amostral:
\begin{equation}
	\hat{V}_{\textbf{x}\textbf{x}}[k] = \frac{1}{N - k + 1} \sum_{n=k}^{N-1} \kappa(\textbf{x}[n], \textbf{x}[n-k]).
\end{equation}
Este estimador é não-enviesado e assintoticamente consistente~\cite{santamaria-2006}, e foi o utilizado para o cálculo da correntropia nos experimentos deste capítulo.

E por que o nome de correntropia? Uma outra importante característica da função,
omitida neste texto por não ter relevância para o projeto, envolve uma aplicação no
contexto de teoria da informação. Assim, o nome \emph{correntropia} é uma aglutinação
de \emph{correlação} com \emph{entropia}, duas informações que a estatística contém.

\section{Filtro de Correntropia}
\label{section:correntropy:filter}

Estamos finalmente munidos de todas as ferramentas necessárias para deduzirmos o Filtro
de Correntropia. É importante frisar que o Filtro de Wiener FIR causal é empregado em
sua modelagem, e como esse sistema já foi apresentado no Capítulo~\ref{chapter:wiener},
a discussão não será secundada aqui.

Nosso ponto de partida para a dedução do modelo será um teorema envolvendo a função de
correntropia. Dada uma sequência estocástica estacionária $x[n]$ --- por exemplo, o
sinal observado no Filtro de Wiener --- e sua correntropia $V_{xx}[n_1, n_2]$, o
teorema~\cite{pokharel-2006} afirma que existe um mapeamento $z : \mathbb{R} \mapsto
	\mathbb{R}$ tal que
\begin{equation}
	V_{xx}[n_1, n_2] = E \{ z(x[n_1]) z(x[n_2]) \}.
	\label{eq:correntropy:theorem}
\end{equation}
Em outras palavras, a correntropia de $x[n]$ é a autocorrelação da sequência aleatória $z(x[n])$. A prova dessa afirmação, que envolve expandir as funções em autofunções e autovalores, pode ser encontrada em~\cite{pokharel-2006}.

Como vimos na Seção~\ref{section:wiener:fir-filter}, os coeficientes do Filtro de
Wiener são computados por meio da matriz de correlação do sinal de entrada. Então, se
modelarmos o filtro utilizando a sequência $z(x[n])$ como entrada, será possível usar a
correntropia do sinal observado. Desse modo, sejam $x[n]$ o sinal observado e $d[n]$ o
sinal desejado, assim como definido no Capítulo~\ref{chapter:wiener}. Além disso, seja
$L$ o tamanho do filtro. O vetor de entrada $\mathbf{z}[n]$, definido como sendo
\begin{equation}
	\mathbf{z}[n] = \begin{bmatrix}
		z(x[n]) & z(x[n - 1]) & \cdots & z(x[n - L + 1])
	\end{bmatrix}^T,
	\symbl{$\mathbf{z}{[n]}$}{Vetor de entrada do Filtro de Correntropia}
\end{equation}
resulta nos seguintes coeficientes ótimos para o Filtro de Wiener FIR causal:
\begin{equation}
	\mathbf{w}^* = \mathbf{R}_{zz}^{-1} \mathbf{r}_{zd}.
\end{equation}

Com base no teorema apresentado, podemos substituir a matriz $\mathbf{R}_{zz}$ pela
\emph{matriz de correntropia} de $x[n]$. Supondo que o sinal é estacionário no sentido
estrito, a matriz é definida como sendo
\begin{equation}
	\mathbf{V}_{xx} = \begin{bmatrix}
		V_{xx}[0]   & V_{xx}[1]   & \cdots & V_{xx}[L - 1] \\
		V_{xx}[1]   & V_{xx}[0]   & \cdots & V_{xx}[L - 2] \\
		\vdots      & \vdots      & \ddots & \vdots        \\
		V_{xx}[L-1] & V_{xx}[L-2] & \cdots & V_{xx}[0]
	\end{bmatrix}.
	\symbl{$\mathbf{V}_{\cdot\cdot}$}{Matriz de correntropia}
\end{equation}

Assim, a saída do filtro será
\begin{equation}
	\hat{d}[n] = \mathbf{z}[n]^T \mathbf{V}_{xx}^{-1} \mathbf{r}_{zd}.
	\label{eq:correntropy:yn-with-z}
\end{equation}

A Equação~\eqref{eq:correntropy:yn-with-z} apresenta o resultado do filtro em função da
sequência $z(x[n])$, a qual não conhecemos. Para descrevermos a equação puramente em
função de $x[n]$, devemos primeiramente substituir $\mathbf{r}_{zd}$ por um estimador.
Supondo ergodicidade, o vetor pode ser aproximado por sua média temporal,
\begin{equation}
	\hat{\mathbf{r}}_{zd} = \frac{1}{N} \sum_{k=0}^{N-1} d[k] \mathbf{z}[k].
\end{equation}
Então, fazendo a substituição e depois expandindo os produtos vetoriais em~\eqref{eq:correntropy:yn-with-z}, encontramos que
\begin{align}
	\hat{d}[n] & = \mathbf{z}[n]^T \mathbf{V}_{xx}^{-1} \mathbf{r}_{zd}                                                            \\
	           & \simeq \mathbf{z}[n]^T \mathbf{V}_{xx}^{-1} \frac{1}{N} \sum_{k=0}^{N-1} d[k] \mathbf{z}[k]                       \\
	           & = \frac{1}{N} \sum_{k=0}^{N-1} \sum_{i=0}^{L-1} \sum_{j=0}^{L-1} z(x[n-i]) \Tilde{v}_{ij} d[k] z(x[k-j])          \\
	           & = \frac{1}{N} \sum_{k=0}^{N-1} d[k] \sum_{i=0}^{L-1} \sum_{j=0}^{L-1} \Tilde{v}_{ij} z(x[n-i]) z(x[k-j])          \\
	           & \simeq \frac{1}{N} \sum_{k=0}^{N-1} d[k] \sum_{i=0}^{L-1} \sum_{j=0}^{L-1} \Tilde{v}_{ij} \kappa(x[n-i], x[k-j]),
	\label{eq:correntropy:filter}
\end{align}
onde $\Tilde{v}_{ij}$\symbl{$\Tilde{v}_{ij}$}{Elemento na $i$-ésima linha e $j$-ésima coluna do inverso da matriz de correntropia} é o elemento da $i$-ésima linha e $j$-ésima coluna de $\mathbf{V}_{xx}^{-1}$. A expressão final é encontrada aproximando-se $z(x[n-i]) z(x[k-j])$ por $\kappa(x[n-i], x[k-j])$, que é uma aproximação válida na média de acordo com a Equação~\eqref{eq:correntropy:theorem}.

O Filtro de Correntropia é totalmente caracterizado pela
Equação~\eqref{eq:correntropy:filter}; com ela, calcula-se a $n$-ésima amostra da
estimação. A Figura~\ref{fig:correntropy:filter} ilustra este processo, considerando
que o inverso da matriz de correntropia já foi computado, e usando versões vetoriais de
$x[n]$, como definido na Equação~\eqref{eq:wf:xn-vector}. Em~\cite{pokharel-2006,
	pokharel-2007}, menciona-se uma etapa adicional em que se iguala a variância de
$\hat{d}[n]$ com a de $d[n]$ por meio de uma correção de ganho, e que essa disparidade
é muito provavelmente resultado da aproximação feita no último passo da modelagem do
filtro. Nos testes do projeto, foi observado que essa correção é \emph{essencial} para
a obtenção de resultados satisfatórios.
\begin{figure}[!ht]
	\centering
	\begin{subfigure}[t]{\textwidth}
		\centering
		\begin{tikzpicture}[node distance=0.75cm and 2cm]
    % x[0] row
    \node[dspnodeopen, dsp/label=above] (x-0) {$\textbf{x}[0]$};
    \node[dspfilter, below=of x-0, yshift=0.25cm] (block-0) {$S$};
    \draw[dspconn] (x-0) -- (block-0);
    
    \node[dspmixer, right=of block-0] (prod-0) {};
    \node[dspnodeopen, dsp/label=above, above=of prod-0] (d-0) {$d[0]$};
    \draw[dspconn] (d-0) -- (prod-0);
    \draw[dspconn] (block-0) -- (prod-0);

    % x[1] row
    \node[dspnodeopen, dsp/label=above, below=of block-0, yshift=-0.25cm] (x-1) {$\textbf{x}[1]$};
    \node[dspfilter, below=of x-1, yshift=0.25cm] (block-1) {$S$};
    \draw[dspconn] (x-1) -- (block-1);
    
    \node[dspmixer, right=of block-1] (prod-1) {};
    \node[dspnodeopen, dsp/label=above, above=of prod-1] (d-1) {$d[1]$};
    \draw[dspconn] (d-1) -- (prod-1);
    \draw[dspconn] (block-1) -- (prod-1);
    
    \node[below=of block-1, yshift=0.75cm] (vdots-blocks) {$\vdots$};
    \node[below=of prod-1, yshift=0.47cm] (vdots-prods) {$\vdots$};
    
    % x[N-1] row
    \node[dspnodeopen, dsp/label=above, below=of vdots-blocks, yshift=0cm] (x-N) {$\textbf{x}[N-1]$};
    \node[dspfilter, below=of x-N, yshift=0.25cm] (block-N) {$S$};
    \draw[dspconn] (x-N) -- (block-N);

    \node[dspmixer, right=of block-N] (prod-N) {};
    \node[dspnodeopen, dsp/label=above, above=of prod-N] (d-N) {$d[N-1]$};
    \draw[dspconn] (d-N) -- (prod-N);
    \draw[dspconn] (block-N) -- (prod-N);

    % Connections and other stuff
    \node[dspadder, right=of prod-1] (adder) {};
    \node[dspmixer, right=of adder, xshift=-1cm] (div) {};
    
    \node[dspnodeopen, dsp/label=left, left=of block-1] (x-n) {$\textbf{x}[n]$};
    \node[coordinate, left=of block-0, xshift=2cm] (xcord-0) {};
    \node[coordinate, left=of block-N, xshift=2cm] (xcord-N) {};
    \draw[dspconn] (x-n) -- (xcord-0);
    \draw[dspconn] (x-n) -- (block-1);
    \draw[dspconn] (x-n) -- (xcord-N);

    \draw[dspconn] (prod-0) -- (adder);
    \draw[dspconn] (prod-1) -- (adder);
    \draw[dspconn] (prod-N) -- (adder);
    
    \draw[dspconn] (adder) -- (div);
    \node[dspnodeopen, dsp/label=left, above=of div, yshift=-0.25cm] (div-N) {};
    \node[yshift=-0.75cm, above=of div-N] (div-N-label) {$\displaystyle\frac{1}{N}$};
    \node[dspnodeopen, dsp/label=above, right=of div, xshift=-1cm] (y-n) {$\hat{d}[n]$};
    \draw[dspconn] (div) -- (y-n);
    \draw[dspconn] (div-N) -- (div);
\end{tikzpicture}
		\caption{Diagrama de blocos para o cálculo da $n$-ésima amostra.}
		\label{fig:correntropy:block-diagram}
	\end{subfigure}

	\bigskip\medskip

	\begin{subfigure}[t]{\textwidth}
		\centering
		\begin{tikzpicture}[node distance=0.75cm and 1cm]
    \node[dspfilter, minimum height=2cm, minimum width=9cm, text height=0.3cm] (sigma) {
        \footnotesize
        \begingroup
        \setlength\arraycolsep{5pt}
        $\begin{matrix}
        \Tilde{v}_{00} \kappa(x[n], x[k]) & \cdots & \Tilde{v}_{0L-1} \kappa(x[n], x[k-L+1])\\
        \vdots & \ddots & \vdots\\
        \Tilde{v}_{L-10} \kappa(x[n-L+1], x[k]) & \cdots & \Tilde{v}_{L-1L-1} \kappa(x[n-L+1], x[k-L+1])
        \end{matrix}$
        \endgroup
    };

    \node[dspnodeopen, dsp/label=above, left=of sigma] (x-n) {$\textbf{x}[n]$};
    \node[dspnodeopen, dsp/label=left, above=of sigma] (x-i) {$\textbf{x}[k]$};

    \draw[dspconn] (x-n) -- (sigma);
    \draw[dspconn] (x-i) -- (sigma);
    \node[coordinate, right=of sigma, xshift=1cm, inner sep=0pt] (sum-cord) {};
    \draw[dspconn] (sigma) -- node[midway,above] {$\sum_i \sum_j s_{ij}$} (sum-cord);
\end{tikzpicture}

		\caption{Detalhamento do bloco $S$ do diagrama.}
		\label{fig:correntropy:sigma-specification}
	\end{subfigure}

	\caption[Diagrama de blocos do Filtro de Correntropia]{Diagrama de blocos que ilustra o algoritmo do Filtro de Correntropia, e detalhe do bloco $S$; $\mathbf{x}[k]$ é um vetor como definido em~\eqref{eq:wf:xn-vector}. Ilustrações baseadas em desenhos similares presentes em~\cite{pokharel-2007}.}
	\label{fig:correntropy:filter}
\end{figure}

\subsection{Solução matricial}

A Equação~\eqref{eq:correntropy:filter} apresenta um problema operacional grave: a
complexidade para se calcular uma amostra da saída, mesmo com o inverso da matriz de
correntropia já computado, é $O(N L^2)$; consequentemente, a complexidade para gerarmos
a estimação em sua totalidade é $O(N^2 L^2)$. Isso ocorre porque não temos coeficientes
explícitos associados ao filtro, diferentemente do Filtro de Wiener. Para uma gravação
de áudio de alta qualidade, que normalmente contém mais de 40 mil amostras por segundo,
uma complexidade tão alta se torna um obstáculo para o desenvolvimento do projeto.

Para mitigar esse problema, podemos tentar reescrever a
Equação~\eqref{eq:correntropy:filter} como uma expressão matricial. Assim, embora o
algoritmo continue sendo de alta complexidade, ele poderá ser vetorizado, resultando
num ganho computacional considerável em máquinas que suportem esse tipo de operação.

Começaremos definindo a \emph{matriz núcleo} de $x[n]$, a qual será representada por
$\mathbf{K}_{xx}$. Esta matriz é quadrada, simétrica, com dimensões $N \times N$ (ou
seja, sem as condições iniciais nulas de $x[n]$), e seus elementos são definidos por
\begin{equation}
	(\mathbf{K}_{xx})_{ij} = \kappa(x[N - 1 - i], x[N - 1 - j]).
	\symbl{$\mathbf{K}_{\cdot\cdot}$}{Matriz núcleo}
\end{equation}

Com a matriz núcleo, é possível computar todos os somatórios associados aos índices $i$
e $j$ na Equação~\eqref{eq:correntropy:filter} em ``uma operação só'', realizando uma
convolução matricial\footnote{Normalmente, em uma convolução matricial, uma das
	matrizes tem suas linhas e colunas invertidas antes do processo, de modo similar ao que
	acontece no caso funcional. Na nossa aplicação, porém, essa inversão não é aplicada.}
entre $\textbf{V}_{xx}^{-1}$ e $\mathbf{K}_{xx}$. Em outras palavras,
\begin{equation}
	(\textbf{V}_{xx}^{-1} * \mathbf{K}_{xx})_{ij} = \sum_{p=i}^{i+L-1} \sum_{q=j}^{j+L-1} \Tilde{v}_{(p-i)(q-j)} \kappa(x[N - 1 - p], x[N - 1 - q]).
\end{equation}
Antes da convolução, é feito um \textit{zero padding} assimétrico em $\mathbf{K}_{xx}$, adicionando $L-1$ linhas nulas abaixo e $L-1$ colunas nulas à direita da matriz. Assim, o resultado da convolução é uma matriz de dimensões $N \times N$.

Finalmente, definindo $\mathbf{d} \in \mathbb{R}^N$ como sendo o vetor com todas as
amostras do sinal desejado, ou seja,
\begin{equation}
	\mathbf{d} = \begin{bmatrix}
		d[N-1] & d[N-2] & \cdots & d[1] & d[0]
	\end{bmatrix}^T,
	\symbl{$\mathbf{d}$}{Vetor com todas as amostras do sinal desejado}
\end{equation}
o Filtro de Correntropia pode então ser representado da seguinte forma:
\begin{equation}
	\hat{\mathbf{d}} = \frac{1}{N} (\textbf{V}_{xx}^{-1} * \mathbf{K}_{xx}) \mathbf{d}.
\end{equation}

\section{Testes preliminares}

Se comparado a outros métodos mais consolidados --- como o próprio Filtro de Wiener,
por exemplo ---, o Filtro de Correntropia é um modelo ainda pouco explorado. Isto é
corroborado pela literatura consultada para o projeto, na qual não há uma discussão
suficientemente satisfatória acerca das capacidades e limitações do algoritmo. Por esse
motivo, é interessante que façamos uma análise preliminar do filtro; assim,
conseguiremos identificar suas particularidades, o que idealmente facilitará não
somente a modelagem do sistema de estimação, como também a interpretação dos resultados
dos experimentos.

Em nossos testes, usaremos dois sinais principais; ambos mais simples do que as
gravações sonoras utilizadas nos experimentos finais, mas não menos significativos. O
primeiro é composto pelo somatório de cinco senoides, uma delas assumindo a frequência
fundamental de 3~Hz, e as restantes representando os quatro harmônicos subsequentes. A
forma de onda resultante, a qual foi amostrada a uma frequência de $f_s =$ 2~kHz
durante 1~s, se encontra na Figura~\ref{fig:correntropy:sinewave}. Por se tratar de um
sinal suave, será possível avaliar graficamente a eficiência do método, sem precisarmos
nos basear apenas em medidas numéricas.

\begin{figure}[!ht]
	\centering
	\begin{tikzpicture}
    \begin{axis}[
        xmin=0, xmax=1,
        ymin=-1, ymax=1,
        xtick={0,0.25,0.5,0.75,1},
        ytick={-1,-0.5,0,0.5,1},
        grid=major,
        width=\linewidth,
        height=7cm
    ]

        \addplot[
            color=black,
            very thick
        ] plot table {data/correntropy-sinewave.dat};

    \end{axis}
\end{tikzpicture}

	\caption[Forma de onda gerada a partir de senoides utilizada nos experimentos preliminares do Filtro de Correntropia]{Forma de onda utilizada nos experimentos preliminares do Filtro de Correntropia, gerada a partir de senoides. Uma dessas senoides está na frequência fundamental de 3~Hz, e as outras assumem os quatro harmônicos consecutivos: 6~Hz, 9~Hz, 12~Hz e 15~Hz. As amplitudes dos componentes foram omitidas aqui por falta de relevância, mas podem ser encontradas no código do projeto~\cite{nonlinear-filters-repo}.}
	\label{fig:correntropy:sinewave}
\end{figure}

O segundo sinal a ser utilizado é um modelo autorregressivo
(AR\abbrev{AR}{Autorregressivo}). Uma forma comum de se modelar sinais é por meio de um
sistema fonte-filtro, em que a entrada é um ruído branco e a resposta em frequência do
filtro representa o espectro do sinal desejado. O modelo AR é baseado em um filtro do
tipo só-polos, no qual as raízes do denominador definem o espectro. Este sistema é
muito usado em processamento de áudio porque a maioria das emissões sonoras envolve
ressonâncias --- ou seja, polos~\cite{godsill-2002}. Deste modo, ao usarmos este modelo
nos testes, poderemos ter uma primeira impressão de como o algoritmo se comporta com
gravações sonoras. O sinal foi gerado com um filtro de ordem 80 (40 pares de polos), a
uma taxa de amostragem de $f_s = 44.1$~kHz durante 200~ms, e sua faixa de valores foi
limitada em $[-1, 1]$, assim como em sinais de áudio digitais.

Como o modelo autorregressivo resulta em uma forma de onda de aparência ruidosa, não
será possível avaliar a eficácia do filtro graficamente como faremos com a entrada
senoidal. Por isso, metrificaremos numericamente a qualidade da estimação com a SDR,
que pode ser prontamente generalizada para quaisquer dois sinais unidimensionais. Além
disso, também tentaremos estimar os sinais com o Filtro de Wiener, para que possamos
observar se o Filtro de Correntropia é melhor, pior, ou tão bom quanto um filtro linear
ótimo (para as condições propostas).

\subsection{Experimento: estimação linear}

O primeiro experimento visou a avaliar a capacidade do filtro de reproduzir distorções
lineares. Para isso, os sinais de teste foram ambos filtrados pelo mesmo passa-baixas:
uma média móvel de ordem 150. Porém, para tornar o experimento mais interessante, foi
estipulado que ambos os filtros testados tivessem ordem $L = 10$, um valor
substancialmente menor do que a ordem do sistema a ser reproduzido. O valor considerado
ótimo para o parâmetro do núcleo gaussiano foi $\sigma = 1 \times 10^{-10}$. Os
resultados do experimento se encontram na
Tabela~\ref{tab:correntropy:poc-experiment-1}; na
Figura~\ref{fig:correntropy:poc-experiment-1}, encontram-se as formas de onda estimadas
pelos filtros para o sinal senoidal. {\def\arraystretch{1.25}\tabcolsep=10pt
		\begin{table}[!ht]
			\centering
			\caption[Resultados do primeiro experimento preliminar: estimação linear]{Resultados, em SDR, do primeiro experimento preliminar.}
			\label{tab:correntropy:poc-experiment-1}
			\begin{tabular}{cccc}
				\toprule
				          & Original    & Wiener    & Correntropia \\ \midrule
				Senoides  & $-2.99$ dB  & $3.10$ dB & $41.06$ dB   \\
				Modelo AR & $-46.30$ dB & $3.24$ dB & $75.54$ dB   \\ \bottomrule
			\end{tabular}
		\end{table}
	}
\begin{figure}[!ht]
	\centering
	\begin{tikzpicture}
    \begin{axis}[
        xmin=0, xmax=1,
        ymin=-1, ymax=1,
        xtick={0,0.25,0.5,0.75,1},
        ytick={-1,-0.5,0,0.5,1},
        grid=major,
        width=\linewidth,
        height=7cm,
        legend style={font=\footnotesize, at={(0.98,0.95)}, anchor=north east}
    ]

        \addplot[
            color=lightgray,
            very thick,
            densely dotted
        ] plot table {data/correntropy-sinewave.dat};
        \addlegendentry{Original}

        \addplot[
            color=black,
            very thick
        ] plot table[x index=0, y index=1] {data/correntropy-poc-experiment-1.dat};
        \addlegendentry{Desejado}

        \addplot[
            color=gray,
            very thick
        ] plot table[x index=0, y index=2] {data/correntropy-poc-experiment-1.dat};
        \addlegendentry{Wiener}

        \addplot[
            color=lightgray,
            very thick,
            densely dashed
        ] plot table[x index=0, y index=3] {data/correntropy-poc-experiment-1.dat};
        \addlegendentry{Correntropia}
    \end{axis}
\end{tikzpicture}

	\caption[Saída desejada e estimações do primeiro experimento preliminar]{Gráfico com a saída desejada (quando a entrada é o somatório de senoides) e as estimações computadas pelos filtros para o primeiro experimento preliminar com o Filtro de Correntropia.}
	\label{fig:correntropy:poc-experiment-1}
\end{figure}

Como podemos observar, o Filtro de Correntropia conseguiu estimar melhor as saídas
desejadas, mesmo com ordem reduzida. Dois fatores possivelmente explicam esse fenômeno:
em primeiro lugar, tem-se o mapeamento infinito realizado pelo núcleo gaussiano, o que
pode ter compensado a menor ordem; além disso, o filtro pode ter feito melhor proveito
das amostras do sinal desejado, já que estas são usadas diretamente na geração das
amostras da estimação (ao contrário do Filtro de Wiener, que as usa somente no cálculo
do vetor de correlação cruzada).

\subsection{Experimento: estimação não-linear}

Neste experimento, queremos observar a capacidade de estimação de distorções
não-lineares do filtro, o principal motivador deste capítulo. Para isso, os sinais de
teste foram processados por uma função de \textit{soft clipping} definida por
\begin{equation}
	f(x) = \frac{2}{\pi} \arctan(1.5 x),
\end{equation}
a qual apresenta comportamento aproximadamente linear em torno da origem, e comprime sinais na faixa de $[-1, 1]$ para a faixa de $[-0.63, 0.63]$. A ordem definida para o Filtro de Wiener foi de $L = 34$, ao passo que para o Filtro de Correntropia a ordem com os melhores resultados foi de $L = 1$, em conjunto com um desvio de $\sigma = 0.9$. As notas se encontram na Tabela~\ref{tab:correntropy:poc-experiment-2}; na Figura~\ref{fig:correntropy:poc-experiment-2}, as formas de onda estimadas para o sinal senoidal são apresentadas.
	{\def\arraystretch{1.25}\tabcolsep=10pt
		\begin{table}[!ht]
			\centering
			\caption[Resultados do segundo experimento preliminar: estimação não-linear]{Resultados, em SDR, do segundo experimento preliminar.}
			\label{tab:correntropy:poc-experiment-2}
			\begin{tabular}{cccc}
				\toprule
				          & Original   & Wiener     & Correntropia \\ \midrule
				Senoides  & $8.72$ dB  & $20.05$ dB & $33.59$ dB   \\
				Modelo AR & $11.96$ dB & $20.14$ dB & $31.59$ dB   \\ \bottomrule
			\end{tabular}
		\end{table}
	}
\begin{figure}[!ht]
	\centering
	\begin{tikzpicture}
    \begin{axis}[
        xmin=0, xmax=1,
        ymin=-1, ymax=1,
        xtick={0,0.25,0.5,0.75,1},
        ytick={-1,-0.5,0,0.5,1},
        grid=major,
        width=\linewidth,
        height=7cm,
        legend style={font=\footnotesize, at={(0.98,0.95)}, anchor=north east}
    ]

        \addplot[
            color=lightgray,
            very thick,
            densely dotted
        ] plot table {data/correntropy-sinewave.dat};
        \addlegendentry{Original}

        \addplot[
            color=black,
            very thick
        ] plot table[x index=0, y index=1] {data/correntropy-poc-experiment-2.dat};
        \addlegendentry{Desejado}

        \addplot[
            color=gray,
            very thick
        ] plot table[x index=0, y index=2] {data/correntropy-poc-experiment-2.dat};
        \addlegendentry{Wiener}

        \addplot[
            color=lightgray,
            very thick,
            densely dashed
        ] plot table[x index=0, y index=3] {data/correntropy-poc-experiment-2.dat};
        \addlegendentry{Correntropia}
    \end{axis}
\end{tikzpicture}

	\caption[Saída desejada e estimações do segundo experimento preliminar]{Gráfico com a saída desejada (quando a entrada é o somatório de senoides) e as estimações computadas pelos filtros para o segundo experimento preliminar com o Filtro de Correntropia.}
	\label{fig:correntropy:poc-experiment-2}
\end{figure}

Os resultados ficaram bem próximos. Mas isso era esperado; afinal, a função de
\textit{soft clipping} afetou apenas os máximos e mínimos locais mais acentuados dos
sinais. Porém, é possível ver claramente que o Filtro de Correntropia gerou as melhores
estimações: note, na Figura~\ref{fig:correntropy:poc-experiment-2}, como o método
conseguiu reproduzir a limitação nos picos e vales, o que o Filtro de Wiener foi
incapaz de fazer.

\subsection{Experimento: robustez a ruído}

A última característica de grande relevância para o projeto é a robustez do filtro a
sinais espúrios. Estes podem ser introduzidos no sistema em dois pontos distintos:
antes ou depois da aplicação da distorção. Ambos os casos são importantes e envolvem um
conjunto de particularidades próprias; porém, como os resultados observados foram
similares, focaremos apenas no caso de adição após a distorção.

Neste par de experimentos, foram usados os mesmos sinais desejados da subseção
anterior, mas agora com ruído aditivo nas observações. Às senoides, somou-se ruído
branco gaussiano com $-20$~dB de potência; já ao processo autorregressivo, foi somado
outro modelo AR gerado de forma idêntica (um ganho de $0.5$ foi aplicado à mistura para
manter a limitação de amplitude). Foram mantidos os parâmetros de $L = 34$ para o
Filtro de Wiener e $L = 1$, $\sigma = 0.9$ para o Filtro de Correntropia. Os resultados
estão presentes na Tabela~\ref{tab:correntropy:poc-experiment-3}; na
Figura~\ref{fig:correntropy:poc-experiment-3} encontram-se as estimações para a entrada
senoidal. {\def\arraystretch{1.25}\tabcolsep=10pt
		\begin{table}[!ht]
			\centering
			\caption[Resultados do terceiro experimento preliminar: robustez a ruído]{Resultados, em SDR, do terceiro experimento preliminar.}
			\label{tab:correntropy:poc-experiment-3}
			\begin{tabular}{cccc}
				\toprule
				          & Original   & Wiener     & Correntropia \\ \midrule
				Senoides  & $8.72$ dB  & $19.67$ dB & $24.83$ dB   \\
				Modelo AR & $11.96$ dB & $6.01$ dB  & $9.51$ dB    \\ \bottomrule
			\end{tabular}
		\end{table}
	}
\begin{figure}[!ht]
	\centering
	\begin{tikzpicture}
    \begin{axis}[
        xmin=0, xmax=1,
        ymin=-1, ymax=1,
        xtick={0,0.25,0.5,0.75,1},
        ytick={-1,-0.5,0,0.5,1},
        grid=major,
        width=\linewidth,
        height=7cm,
        legend style={font=\footnotesize, at={(0.98,0.95)}, anchor=north east}
    ]

        \addplot[
            color=lightgray,
            very thick,
            densely dotted
        ] plot table {data/correntropy-sinewave.dat};
        \addlegendentry{Original}

        \addplot[
            color=black,
            very thick
        ] plot table[x index=0, y index=1] {data/correntropy-poc-experiment-3.dat};
        \addlegendentry{Desejado}

        \addplot[
            color=gray,
            very thick
        ] plot table[x index=0, y index=2] {data/correntropy-poc-experiment-3.dat};
        \addlegendentry{Wiener}

        \addplot[
            color=lightgray,
            very thick,
            densely dashed
        ] plot table[x index=0, y index=3] {data/correntropy-poc-experiment-3.dat};
        \addlegendentry{Correntropia}
    \end{axis}
\end{tikzpicture}
	\caption[Saída desejada e estimações do terceiro experimento preliminar]{Gráfico com a saída desejada (quando a entrada é o somatório de senoides) e as estimações computadas pelos filtros para o terceiro experimento preliminar com o Filtro de Correntropia.}
	\label{fig:correntropy:poc-experiment-3}
\end{figure}

É possível constatar que os métodos se comportaram de forma similar: no sinal senoidal, ambos os filtros removeram o ruído, e o Filtro de Correntropia continuou conseguindo estimar aproximadamente a distorção; já no modelo AR, nenhum dos dois conseguiu replicar o sinal desejado como no experimento anterior. Foi constatada também uma maior sensibilidade do Filtro de Correntropia a ganhos aplicados no sinal observado: sem a correção de ganho de $0.5$ na mistura de sinais autorregressivos, a estimação do filtro caía para torno de $0$~dB.

\subsection{Considerações adicionais}
\label{subsec:correntropy:considerations}

Antes de continuarmos, algumas considerações comuns a todos os experimentos devem ser
mencionadas. Primeiramente, em~\cite{pokharel-2006} os autores recomendam que o valor
do parâmetro $\sigma$ seja em torno de $15\%$ do desvio-padrão do sinal de entrada;
porém, os desvios-padrão do sinal senoidal e do modelo AR foram, respectivamente,
$0.45$ e $0.37$. Com $\sigma = 1 \times 10^{-10}$ para o primeiro experimento e $\sigma
	= 0.9$ para os restantes, observa-se que os valores ótimos usados se encontram muito
distantes do recomendado. Deste modo, a regra geral dos $15\%$ não se mostrou muito
efetiva, e a abordagem tomada (não só nessa seção, como também na
Seção~\ref{section:correntropy:experiments}) foi a de testar diferentes valores do
parâmetro até encontrarmos o melhor resultado possível.

Outro ponto que deve ser reiterado é a importância da correção de ganho aplicada ao fim
do algoritmo. Sem ela, o sinal estimado apresenta amplitudes consideravelmente menores
do que o sinal desejado, embora se comporte de modo similar. Esse ajuste é também
responsável por um problema comum no filtro, em que a estimação resultante aparenta
estar com ruído aditivo. Quando removida a correção de ganho, as amostras dos sinais
que apresentaram esse fenômeno encontravam-se todas muito próximas de zero; ou seja, a
correção amplificou erros numéricos. Caso isso ocorra, é recomendado que ou $L$ ou
$\sigma$ sejam incrementados.

Finalmente, um último detalhe envolve a escolha conjunta dos valores de $L$ e $\sigma$.
Após os testes, foi constatado que aumentar indiscriminadamente a ordem do filtro não é
recomendado; no geral, deve-se priorizar valores pequenos, até mesmo $L = 1$, e
incrementar a ordem aos poucos conforme a necessidade. Ademais, escolher um desvio
$\sigma$ grande em conjunto a um $L$ alto pode resultar em amostras aberrantes no
início da estimação, afetando seu desvio-padrão e, consequentemente, a correção de
ganho (em casos extremos, os \textit{outliers} acabam fazendo com que o fator aplicado
praticamente zere as outras amostras). Um modo de diagnosticar um par inadequado de
parâmetros é computando o número de condicionamento da matriz de correntropia
$\mathbf{V}_{xx}$: foi observado que matrizes com número de condicionamento maior que
$10^3$ normalmente apresentavam maus resultados, salvo raras exceções.

\section{Método proposto}

Assim como no caso do Filtro de Wiener, foi necessário introduzir algumas etapas
adicionais ao algoritmo para possibilitar a aplicação do Filtro de Correntropia no
contexto do trabalho. O problema da não-estacionariedade de sinais de áudio foi
novamente mitigado por meio da técnica de \textit{overlap-add}. Além disso, o desvio
$\sigma$ não foi ajustado manualmente, mas sim variado de modo adaptativo: a cada bloco
processado, o valor do parâmetro é definido de modo a otimizar a estimação local --- ou
seja, do bloco --- de acordo com uma medida escolhida (ou seja, uma decisão de
projeto). Essa abordagem emergiu como alternativa à opção de manter o desvio constante
durante todo o processamento, estratégia que resultou em estimações ruidosas e errôneas
para todos os valores de $L$ e $\sigma$ experimentados.

\begin{figure}[!ht]
	\centering
	\begin{tikzpicture}[node distance=1.25cm and 1cm]
    \draw[dspconn] (-2, -2) -- (2, 1.5);

    % Filters
    \node[dspfilter, minimum width=4cm, minimum height=1.5cm, fill=white] (corr) {$\frac{1}{M} (\textbf{V}_{x_i x_i}^{-1} * \mathbf{K}_{x_i x_i}) \mathbf{y}_i$};
    \node[dspfilter, minimum width=4cm, minimum height=1.5cm, below=of corr, yshift=0.5cm, fill=white] (algorithm) {Variador de $\sigma$};

    % nodes
    \node[dspnodeopen, dsp/label=left, above=of corr] (x) {$x_i[n]$};
    \node[dspnodefull, dsp/label=above, left=of corr] (mid-d) {};
    \node[dspnodeopen, dsp/label=above, left=of mid-d] (d) {$y_i[n]$};
    \node[dspnodefull, right=of corr, dsp/label=above] (mid-corr) {$\hat{d}_i[n]$};

    % Window
    \node[dspfilter, minimum width=2.5cm, minimum height=1.25cm, right=of mid-corr] (win) {};
    \pic at (win) {window};

    \draw[dspconn] (x) -- (corr);
    \draw[dspconn] (d) -- (corr);
    \draw[dspconn] (corr) -- (win);
    \draw[dspconn] (mid-corr) |- (algorithm);
    \draw[dspconn] (mid-d) |- (algorithm);
    
	\node[dspnodeopen, dsp/label=left, below right=of win, yshift=0.75cm, xshift=0.25cm] (ola-pic) {$\hat{d}[n]$};
	\draw[dspconn] (win) -| node[midway,below, xshift=0.3cm, yshift=-0.25cm] {$+$} (ola-pic);
    \pic[xshift=-2cm, yshift=-1.5cm] at (ola-pic) {ola};
\end{tikzpicture}
	\caption[Ilustração do método utilizado para o Filtro de Correntropia]{Ilustração do método de \textit{overlap-add} com $\sigma$ adaptativo, utilizado na implementação prática do Filtro de Correntropia, para a $i$-ésima iteração.}
	\label{fig:correntropy:method}
\end{figure}

A Figura~\ref{fig:correntropy:method} ilustra o método em sua $i$-ésima iteração. Os
sinais $x_i[n]$ e $y_i[n]$ são sequências de comprimento $M$ (o tamanho da janela) com
as amostras da gravação não-distorcida e da observação, respectivamente, a serem
usadas. Com estes sinais (e as $L - 1$ amostras anteriores de $x[n]$, utilizadas como
condições iniciais), são computadas várias instâncias do bloco
$\hat{d}_i[n]$\symbl{$\hat{d}_i{[n]}$}{Sinal desejado estimado na $i$-ésima iteração do
método de estimação}, de tamanho $M$, com o Filtro de Correntropia, cada uma com um
desvio $\sigma$ diferente. A instância que resultar no melhor valor para uma medida
estipulada é usada na geração de $\hat{d}[n]$. Ademais, é possível aplicar um
janelamento antes da adição de $\hat{d}_i[n]$ na estimação final; só é preciso que a
escolha da janela e do salto entre iterações $H$ seja feita com cuidado, de modo a
manter a condição de \textit{constant overlap-add}. Como todo o processamento é feito
no domínio temporal, não foi aplicado nenhum janelamento de pré-processamento (além do
retangular no instante da extração do bloco).

\section{Experimentos}
\label{section:correntropy:experiments}

Os experimentos a seguir foram executados com as mesmas faixas de áudio da
Seção~\ref{section:wf:experiments}: a gravação da flauta hulusi como sinal original
$x[n]$, e o diálogo de \textit{Mansfield Park} como ruído $r[n]$; ambas com os mesmos
pré-condicionamentos anteriormente especificados (mono, taxa de amostragem de
$44.1$~kHz, \textit{bitdepth} de 16~bits, e ganho de normalização para 0~dB). O leitor
que queira refrescar a memória é encorajado a revisitar as
Figuras~\ref{fig:wf:spectogram-reference} e~\ref{fig:wf:spectogram-dialogue}, as quais
apresentam os espectrogramas dos sinais. Os experimentos feitos foram os mesmos da
Seção~\ref{section:wf:experiments}, e as condições de cada um serão brevemente
recapituladas por questão de conveniência.

A janela escolhida para o processo de síntese foi a função de Hann com $M$ par e
sobreposição de $50\%$ ($H = M/2$), definida como
\begin{equation}
	h[n] = \frac{1}{2} - \frac{1}{2}\cos\left(\frac{2\pi n}{M}\right) = \sen^2\left( \frac{\pi n}{M} \right),\ 0 \leq n \leq M-1.
\end{equation}
Nestas condições, o \textit{overlap-add} é constante e de ganho unitário; assim, nenhum ganho de pós-processamento precisou ser aplicado.

Os experimentos foram todos executados com um filtro de tamanho $L = 1$, o qual
apresentou os melhores resultados. A cada iteração, foram testados 250 valores
diferentes de $\sigma$, com variação de $10^{-9}$ a $10$ em escala logarítmica, e o
bloco que resultasse na maior SDR entre $\hat{d}_i[n]$ e $y_i[n]$ era o escolhido. O
tamanho da janela, estipulado como $M = 8820$, foi um meio-termo entre eficiência e
complexidade computacional: intervalos maiores tiveram resultados melhores, mas apenas
marginalmente, de tal modo que o custo computacional (trabalhar com matrizes com mais
de 100 milhões de elementos) não justificou o aumento.

\subsection{\textit{Fades}}

Neste primeiro experimento, avaliaremos a capacidade do método em estimar um sistema
linear variante no tempo. Para isso, aplicou-se no início da gravação original um
\textit{fade-out} com parâmetros $A_\text{i} = 1$ e $A_\text{f} = 0.2$, e um
\textit{fade-in} com parâmetros $A_\text{i} = 0.2$ e $A_\text{f} = 1$ em seu fim. Cada
efeito, que seguiu uma curva exponencial, durou aproximadamente 1.5 segundo. Os
resultados encontram-se na Tabela~\ref{tab:correntropy:experiment-1}.
{\def\arraystretch{1.25}\tabcolsep=10pt
\begin{table}[!ht]
	\centering
	\caption[Resultados do primeiro experimento: \textit{fades}]{Resultados do primeiro experimento principal.}
	\label{tab:correntropy:experiment-1}
	\begin{tabular}{cccc}
		\toprule
		               & SDR         & PAQM (MOS) & $\log_{10}(R_{\text{nonlin}})$ \\
		\midrule
		$(d, x)$       & $-11.57$ dB & $1.17$     & $-0.005$                       \\
		$(d, \hat{d})$ & $21.54$ dB  & $1.18$     & $-0.019$                       \\ \bottomrule
	\end{tabular}
\end{table}
}

Houve uma considerável melhoria na SDR, embora não tanto quanto como no Filtro de
Wiener. Além disso, podemos notar uma pequena variação nas outras duas medidas. Ouvindo
a gravação estimada, percebe-se que o método introduziu um efeito similar ao de
limitação de amplitude em momentos de maior intensidade do sinal, o que explica o
decréscimo de $\log_{10}(R_{\text{nonlin}})$. Gerando o espectrograma de $\hat{d}[n]$,
foi possível observar a presença de harmônicos nesses instantes.

\subsection{\textit{Fades} com ruído aditivo}

Neste experimento, foi usada exatamente a mesma gravação distorcida do experimento
anterior, porém agora com ruído aditivo introduzido no intervalo em que seu ganho
estava em $A = 0.2$. É importante reiterar que a faixa de diálogo é consideravelmente
mais intensa do que o sinal desejado, o que pode dificultar a estimação. A SNR da
observação é $-9.17$~dB. Os resultados encontram-se na
Tabela~\ref{tab:correntropy:experiment-2}. {\def\arraystretch{1.25}\tabcolsep=10pt
\begin{table}[!ht]
	\centering
	\caption[Resultados do segundo experimento: \textit{fades} com ruído aditivo]{Resultados do segundo experimento principal.}
	\label{tab:correntropy:experiment-2}
	\begin{tabular}{cccc}
		\toprule
		               & SDR         & PAQM (MOS) & $\log_{10}(R_{\text{nonlin}})$ \\
		\midrule
		$(d, x)$       & $-11.57$ dB & $1.17$     & $-0.005$                       \\
		$(d, \hat{d})$ & $-8.92$ dB  & $1.17$     & $-0.282$                       \\ \bottomrule
	\end{tabular}
\end{table}
}

As notas insatisfatórias são prontamente fundamentadas ao ouvirmos a estimação gerada:
o ruído se manteve presente na saída, embora com menos intensidade. Isto ocorreu porque
o algoritmo de escolha de $\sigma$ compara $\hat{d}_i[n]$ com $y_i[n]$, logo, blocos
que preservaram o ruído aditivo resultaram em uma SDR maior.\footnote{É verdade que o
	Filtro de Wiener também tenta minimizar a diferença entre a estimação e a observação;
	porém, como o sinal estimado é computado por meio de uma convolução entre os
	coeficientes e a entrada, o filtro é incapaz de introduzir as componentes frequenciais
	adicionais do ruído, limitando a capacidade de permanência deste na estimação.} Isso
não significa, porém, que a culpa é somente do método estipulado: usar diretamente
$d_i[n]$ para a escolha de $\sigma$ (o que seria inviável na prática) também resultou
em uma estimação grosseira, com ganho variável indesejado e artefatos introduzidos.

\subsection{\textit{Soft clipping}}

O terceiro experimento objetivou avaliar a capacidade do filtro em reproduzir o efeito
não-linear de \textit{soft clipping}. Para isso, o sinal original foi processado por
uma função que limitou seu pico máximo de amplitude em $-7.5$~dB. Nesse teste, não foi
introduzido ruído aditivo. Os resultados encontram-se na
Tabela~\ref{tab:correntropy:experiment-3}. {\def\arraystretch{1.25}\tabcolsep=10pt
\begin{table}[!ht]
	\centering
	\caption[Resultados do terceiro experimento: \textit{soft clipping}]{Resultados do terceiro experimento principal.}
	\label{tab:correntropy:experiment-3}
	\begin{tabular}{cccc}
		\toprule
		               & SDR        & PAQM (MOS) & $\log_{10}(R_{\text{nonlin}})$ \\
		\midrule
		$(d, x)$       & $19.56$ dB & $1.21$     & $-0.016$                       \\
		$(d, \hat{d})$ & $17.50$ dB & $1.18$     & $-0.020$                       \\ \bottomrule
	\end{tabular}
\end{table}
}

Temos aqui o primeiro caso em que a aplicação do filtro piorou as três medidas
consideradas. Ao se ouvir a gravação resultante, pode-se perceber que o efeito
proveniente da limitação de amplitude se encontra presente, porém de modo muito mais
agressivo. Deste modo, é possível que tenha ocorrido exatamente o mesmo problema do
primeiro experimento: na tentativa de estimar o processamento, o filtro introduziu
distorções não-lineares indesejadas ao sinal.

\subsection{\textit{Soft clipping} com ruído aditivo}

Queremos agora verificar se o método é capaz de reproduzir o mesmo efeito de
\textit{clipping} do experimento anterior, porém agora com o diálogo introduzido na
observação. Diferentemente do segundo experimento, o sinal desejado possui mais energia
neste: a SNR da observação é $3.97$~dB. Os resultados encontram-se na
Tabela~\ref{tab:correntropy:experiment-4}. {\def\arraystretch{1.25}\tabcolsep=10pt
\begin{table}[!ht]
	\centering
	\caption[Resultados do quarto experimento: \textit{soft clipping} com ruído aditivo]{Resultados do quarto experimento principal.}
	\label{tab:correntropy:experiment-4}
	\begin{tabular}{cccc}
		\toprule
		               & SDR        & PAQM (MOS) & $\log_{10}(R_{\text{nonlin}})$ \\
		\midrule
		$(d, x)$       & $19.56$ dB & $1.21$     & $-0.016$                       \\
		$(d, \hat{d})$ & $4.40$ dB  & $1.17$     & $-0.155$                       \\ \bottomrule
	\end{tabular}
\end{table}
}

Considerando os dois experimentos anteriores, as notas (e o sinal estimado) resultantes
são facilmente justificáveis. Novamente, o algoritmo proposto favoreceu blocos nos
quais o ruído estivesse mais proeminente, pois estes resultavam em altos valores para a
SDR. Porém, os resquícios presentes na estimação não foram tão intensos quanto no
segundo experimento, muito provavelmente por causa da maior energia do sinal desejado.
Ademais, ao ouvirmos a gravação, o efeito de limitação de amplitude não pareceu estar
presente.

\subsection{\textit{Fades} e codificação com perdas}

Este experimento é, de certo modo, uma mistura do primeiro com o terceiro: aplicou-se
um algoritmo de compressão com perdas (distorção não-linear) na versão do sinal com
\textit{fade-out} e \textit{fade-in} (distorção linear). Foi aplicada uma codificação
MP3 com taxa de bits constante de 56 kbps. Os resultados encontram-se na
Tabela~\ref{tab:correntropy:experiment-5}. {\def\arraystretch{1.25}\tabcolsep=10pt
\begin{table}[!ht]
	\centering
	\caption[Resultados do quinto experimento: \textit{fades} e codificação com perdas]{Resultados do quinto experimento principal.}
	\label{tab:correntropy:experiment-5}
	\begin{tabular}{cccc}
		\toprule
		               & SDR         & PAQM (MOS) & $\log_{10}(R_{\text{nonlin}})$ \\
		\midrule
		$(d, x)$       & $-14.77$ dB & $1.17$     & $-0.032$                       \\
		$(d, \hat{d})$ & $5.64$ dB   & $1.17$     & $-0.330$                       \\ \bottomrule
	\end{tabular}
\end{table}
}

Infelizmente, os valores resultantes das medidas não descrevem completamente o que
ocorreu no experimento. Para isso, devemos ouvir a gravação em si, e descobrir que o
método introduziu um ruído de potência extremamente alta na estimação. É possível
perceber também que o filtro reproduziu o efeito dos \textit{fades}. Logo, a distorção
linear foi identificada, mas a não-linear foi ``convertida'' em ruído.

\subsection{\textit{Fades} com ruído aditivo e codificação com perdas}

O último experimento é uma amálgama de todas as condições impostas até agora ao método.
O sinal original recebe a distorção (linear) de \textit{fades}; então, soma-se a faixa
de diálogo (ruído); por fim, a combinação dos dois sinais é codificada (distorção
não-linear). Novamente foi utilizada uma codificação MP3 com taxa de bits constante de
56 kbps. Assim como explicado na Subseção~\ref{subsec:wiener:experiment-6}, o sinal que
usaremos como $d[n]$ \emph{não} é o verdadeiro sinal distorcido presente em $y[n]$,
pois introduzir o ruído antes do processamento não-linear afeta a geração de $d[n]$.
Não foi possível calcular a SNR dessa observação, mas podemos estimar seu limite
superior como sendo $-9.17$~dB. Os resultados encontram-se na
Tabela~\ref{tab:correntropy:experiment-6}. {\def\arraystretch{1.25}\tabcolsep=10pt
\begin{table}[!ht]
	\centering
	\caption[Resultados do sexto experimento: \textit{fades} com ruído aditivo e codificação com perdas]{Resultados do sexto experimento principal.}
	\label{tab:correntropy:experiment-6}
	\begin{tabular}{cccc}
		\toprule
		               & SDR         & PAQM (MOS) & $\log_{10}(R_{\text{nonlin}})$ \\
		\midrule
		$(d, x)$       & $-14.77$ dB & $1.17$     & $-0.032$                       \\
		$(d, \hat{d})$ & $-9.08$ dB  & $1.17$     & $-0.422$                       \\ \bottomrule
	\end{tabular}
\end{table}
}

Curiosamente, a estimação resultante também foi uma amálgama de tudo que observamos nos
experimentos anteriores: o filtro conseguiu reproduzir os efeitos lineares de
\textit{fade-out} e \textit{fade-in}, a codificação com perdas introduziu um ruído de
alta potência na gravação, e a faixa de diálogo não foi completamente removida. Assim,
mesmo que os resultados não tenham sido satisfatórios, eles foram consistentes, e, com
isso, sabemos quais são as capacidades do filtro (com o método proposto) no contexto do
trabalho.
